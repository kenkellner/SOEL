
\begin{table}[htbp]
  \centering
  \caption{Values used to initialize the forest stand in each model run. The initial forest included trees in two size categories (midstory and overstory) for each of the three tree categories in the model: oak (white oak \textit{Quercus alba} and black oak \textit{Q. velutina}), shade tolerant (sugar maple \textit{Acer saccharum}) and shade intolerant (tulip poplar \textit{Liriodendron tulipifera}). Each size $\times$ species category was defined by an initial density (per hectare) and diameter at breast height (dbh) distribution. Values are based on forest structure data collected pre-harvest from the Hardwood Ecosystem Experiment \citep{Saunders2013}.}
    \begin{tabular}{rrccc}
    \toprule
    \textbf{} & \textbf{} & \textbf{} & \multicolumn{2}{c}{\textbf{dbh}} \\
    \multicolumn{1}{l}{\textbf{Species}} & \textbf{Size Category} & \textbf{Trees ha\textsuperscript{-1}} & \textbf{Mean} & \textbf{SD} \\
        \midrule
    \multicolumn{1}{l}{Oaks (both)} & Midstory & 95    & 14.9  & 5 \\
          & Overstory & 89    & 45.75 & 5 \\
    \multicolumn{1}{l}{Sugar maple} & Midstory & 499   & 10.3  & 5 \\
          & Overstory & 11    & 40.8  & 5 \\
    \multicolumn{1}{l}{Tulip poplar} & Midstory & 163   & 14.9  & 5 \\
          & Overstory & 9     & 45.07 & 5 \\
    \bottomrule
    \end{tabular}%
  \label{tab:1}%
\end{table}%
